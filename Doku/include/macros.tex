%-------------------------------------
%-- Informationen für das Deckblatt --
%-------------------------------------
\newcommand{\titel}{\textbf{Studienarbeit}}
\newcommand{\thema}{Real World Integration}
\newcommand{\studiengang}{Informationstechnik}
\newcommand{\dhbw}{Dualen Hochschule Baden-Württemberg Mannheim}
\newcommand{\dlr}{Deutsches Zentrum für Luft- und Raumfahrt e. V.}
\newcommand{\jahrgang}{\textit{MA-TINF13ITIN}}

\newcommand{\praxis}{von hier bis da}
\newcommand{\betreuer}{Prof. Dr. Harald Kornmayer}
\newcommand{\abgabe}{abgabe ist irgendwann}

\newcommand{\autorA}{\textbf{Tobias Haase}}
\newcommand{\matrikelnrA}{4577806}
\newcommand{\autorB}{\textbf{Sven Durchholz}}
\newcommand{\matrikelnrB}{6002482}
\newcommand{\autorC}{\textbf{Phillipp Mevenkamp}}
\newcommand{\matrikelnrC}{9893674}
\newcommand{\autorD}{\textbf{Hendrik Abbenhaus}}
\newcommand{\matrikelnrD}{1234567}

%----------------------------------
%-- Vordefinierte Formatierungen --
%----------------------------------

%format: \setEquation{gleichung}{label for references}
\newcommand{\setEquation}[2]
{
	\begin{figure}[!htbp]
		\centering
		\begin{equation}
			#1
		\end{equation}
		\label{eq:#2}
	\end{figure}
}

%format: \setPicture{path}{description}{label for references}
\newcommand{\setPicture}[3]{
	\begin{figure}[!htbp]
		\centering
		\includegraphics[width=\textwidth]{#1}
		\caption{#2}
		\label{fig:#3}
	\end{figure}
}

%format: \setPicture{path}{width}{description}{label for references}
\newcommand{\setPictureWidth}[4]{
	\begin{figure}[!htbp]
		\centering
		\includegraphics[width=#2]{#1}
		\caption{#3}
		\label{fig:#4}
	\end{figure}
}

%format: \setPictureSourced{path}{width}{description}{label for references}{source}
\newcommand{\setPictureSourced}[5]{
	\begin{figure}[!htb]
		\centering
		\includegraphics[width=#2]{#1}
		\caption[#3]{#3\protect\footnotemark}
		\label{fig:#4}
	\end{figure}
	\footnotetext{Quelle: \citeauthor{#5}, \citetitle{#5}}
}

%format: \setTable{path}{description}{label for references}
\newcommand{\setTable}[3]{
	\begin{figure}[!htbp]
		\centering
		\input{#1}
		\caption{#2}
		\label{#3}
	\end{figure}
}

%format: \refPicture{label for references}
\newcommand{\refPicture}[1]{\figurename~\ref{fig:#1}}
\newcommand{\refTable}[1]{\tablename~\ref{tab:#1}}
\newcommand{\refChapter}[1]{Kapitel \ref{#1}}
\newcommand{\refEquation}[1]{Gleichung \ref{eq:#1}}


% ---------Schriftformatierungen------------------ 
\newcommand{\Name}[1]{\emph{#1}}
\newcommand{\Fachbegriff}[1]{\textbf{#1}} 
\newcommand{\Code}[1]{\texttt{#1}}
\newcommand{\Datei}[1]{\texttt{#1}}
\newcommand{\Datentyp}[1]{\textsf{#1}}

% ---------- Other -----------
\newcommand{\cpp}{C\texttt{++}}
\newcommand{\csh}{C\texttt{\#}}