% --- Farben definieren ---
\definecolor{mygreen}{rgb}{0,0.6,0}
\definecolor{mygray}{rgb}{0.5,0.5,0.5}
\definecolor{mymauve}{rgb}{0.58,0,0.82}

% --- Formattierung Quellcode ---
% https://en.wikibooks.org/wiki/LaTeX/Source_Code_Listings
\lstset{ %
  backgroundcolor=\color{white},   % choose the background color; you must add \usepackage{color} or \usepackage{xcolor}
  basicstyle=\footnotesize,        % the size of the fonts that are used for the code
  breakatwhitespace=false,         % sets if automatic breaks should only happen at whitespace
  breaklines=true,                 % sets automatic line breaking
  captionpos=b,                    % sets the caption-position to bottom
  commentstyle=\color{mygreen},    % comment style
  deletekeywords={...},            % if you want to delete keywords from the given language
  escapeinside={\%*}{*)},          % if you want to add LaTeX within your code
  extendedchars=true,              % lets you use non-ASCII characters; for 8-bits encodings only, does not work with UTF-8
  frame=single,	                   % adds a frame around the code
  keepspaces=true,                 % keeps spaces in text, useful for keeping indentation of code (possibly needs columns=flexible)
  keywordstyle=\color{blue},       % keyword style
  otherkeywords={*,...},           % if you want to add more keywords to the set
  numbers=left,                    % where to put the line-numbers; possible values are (none, left, right)
  numbersep=5pt,                   % how far the line-numbers are from the code
  numberstyle=\tiny\color{mygray}, % the style that is used for the line-numbers
  rulecolor=\color{black},         % if not set, the frame-color may be changed on line-breaks within not-black text (e.g. comments (green here))
  showspaces=false,                % show spaces everywhere adding particular underscores; it overrides 'showstringspaces'
  showstringspaces=false,          % underline spaces within strings only
  showtabs=false,                  % show tabs within strings adding particular underscores
  stepnumber=2,                    % the step between two line-numbers. If it's 1, each line will be numbered
  stringstyle=\color{mymauve},     % string literal style
  tabsize=2,	                   % sets default tabsize to 2 spaces
  title=\lstname                   % show the filename of files included with \lstinputlisting; also try caption instead of title
}

% --- mathtools ---
%\mathtoolsset{showonlyrefs}					% Nummierung von Gleichungen nur, wenn diese einen Verweis besitzen

% --- siunitx ---
\sisetup{
%	mode = math, % text is printed using a math font
	detect-all,
	separate-uncertainty 	= true,				% Plus-Minus-Zeichen für Fehler
	exponent-product 		= \cdot,			% Malzeichen für Exponent
	number-unit-separator 	= \text{\,},		% Komma als Dezimalzeichen statt Punkt
	output-decimal-marker 	= {\text{,}},
	per-mode 				= fraction,			% Darstellung von Brüchen
}

% --- caption ---
%\captionsetup{}

% --- scrlayer-scrpage ---
%\pagestyle{scrheadings}
\clearpairofpagestyles							% Leeren von Kopf- und Fußzeile

%\ohead{\pagemark} 								% Kopfzeile außen: Seitenzahl
\ihead{\headmark} 								% Kopfzeile innen: chapter und section Titel
\cfoot[-~\pagemark~-]{-~\pagemark~-}			% Fußzeile mitte: Seitenzahl
%\setkomafont{pageheadfoot}{}					% Änderung von Kopf- und Fußzeile
%\setkomafont{pagenumber}{}
%\setkomafont{headsepline}{\color{red}}

% --- hyperref ---
\hypersetup{
	pdftitle 			= {Studienarbeit},		% Name des Dokuments
	pdfsubject 			= {Real World Integration},% Thema der Dokuments
	pdfauthor 			= {},				% Autor
%	pdfkeywords 		= {},
%	pdfcreator 			= {},
%	pdfproducer 		= {},
	pdftoolbar 			= true,
	pdfmenubar 			= false,
%	bookmarks 			= true,
	bookmarksopen 		= false,				% bookmarksopen ODER bookmarksopenlevel!
%	bookmarksopenlevel 	= section,
}

% --- biblatex ---
\DefineBibliographyStrings{german}{				% "Datum" für URL in Zitaten
	urlseen = {Besucht:}
}

% ---------------------------
%\KOMAoptions{
%	headlines	= 2								% Für mehrzeilige Überschriften
%	headinclude,								% Überschrift in Satzspiegel berücksichtigen
%	BCOR		= 1cm,							% Bindekorrektur
%	DIV			= calc							% Satzspiegel neuberechnen
%}