% ---------------------------
% Wichtiger Hinweis für die Kodierung:
% 	Inputenc gibt die Kodierung des Dokoments an,
% 	die muss mit der Kodierung der Datei übereinstimmen.
% 	
%   Mögliche Einstellungen:
%	Editor		|	inputenc
%	------------|------------
%	UTF-8		|	utf8			(vorzugsweise verwenden)
%	ISO-8859-1	|	ansinew
%	windows-1252|	ansinew
%	Apple		|	applemac		(von mir nicht geprüft)
%	Linux		| 	latin1, utf8
% ---------------------------

% --- Kodierung, Sprache ---
\usepackage[utf8]{inputenc}						% Kodierung des Dokuments
\usepackage[T1]{fontenc}						% Schriftzeichen, Umlaute, Silbentrennung
\usepackage[ngerman]{babel}						% Neue deutsche Rechtschreibung
\usepackage[printonlyused]{acronym}				%b Abkürzungsverzeichnis

% --- Schrift ---
\usepackage{lmodern}							% Moderne Schriftart
\usepackage{microtype}							% Randausgleich und Neuberechnung der Lücken

% ---------------------------
% Fakulative Packages
% ---------------------------
% Zum Aktivieren Kommentar entfernen
% 
%		---WICHTIG---
% Reihenfolge von einigen Pakete muss beachtet werden, 
% da es ansonsten zu "Option clash" kommen kann!
% ---------------------------

% --- Hacks ---
\usepackage{scrhack}							% Hacks für setspaces, float, hyperref und listings


% --- Tabellen ---
\usepackage{tabularx}							% Verwendung der Tabularx-Umgebung für Tabellen
\usepackage{booktabs}							% Übersichtlichere Darstellung von Tabellen
\usepackage{multirow}							% Zusammenfassen von Tabellenreihen (vertical)


% --- Grafiken ---
\usepackage{graphicx} 							% Einbinden von Bildern
%\usepackage{float}								% U.a neue Option H für figure und table


% --- Formatierungen und Design ---
\usepackage{setspace}							% Ein-, anderhalb- oder zweizeilig
\usepackage[%
	% --- Linienart ---							% Einstellung von Dicke und Breite
%	headtopline,								% durch name = Dicke:Breite
%	plainheadtopline,
	headsepline,
%	plainheadsepline,
%	footsepline,
	plainfootsepline,
%	footbotline,
%	plainfootbotline,
%	ilines,
	clines,
%	olines,
	automark,
%	autooneside = false,						% ignore optional argument in automark at oneside
]{scrlayer-scrpage}								% Kopf- und Fußzeile
\usepackage[
	top 	= 2.5cm,								% Seitenabstand oben
	right 	= 2.5cm,								% Seitenabstand recht			
	bottom 	= 2.5cm,								% Seitenabstand unten
	left	= 2.5cm,								% Seitenabstand links
	includeheadfoot,							% Zitate als Fußnote
	bindingoffset = 1cm							% Bindekorrektur								
]{geometry}
\usepackage[
	table,
	dvipsnames									% Vordefinierte Farben
]{xcolor}										% Farbige Darstellungen
%\usepackage{caption}							% Bilder- u. Tabellenbeschriftung
\usepackage{listings}


% --- Mathematik ---
%\usepackage{cancel} 							% Durchstreichen von Gleichungen
\usepackage{siunitx}							% Darstellung von Maßzahlen und Maßeinheiten
\usepackage[
	intlimits,  								% Grenzen Integral
	%leqno,     								% Nummerierung links
	%reqno,     								% Nummerierung rechts [default]
	%fleqn      								% Formel linksbündig mit festem Abstand
]{amsmath}										% Erweiterung des Mathemodus
\usepackage[
	all,
	warning,
%	error										% Kompilieren von pdfLaTeX kann abgebrochen werden
]{onlyamsmath}									% Anzeigen von Fehlern bei Verwendung von amsmath
%\usepackage[
%	fixamsmath,									% [default]
%	disallowspaces								% Fehlerbehebung
%]{mathtools}									% Erweitung zu amsmath


% --- PDF ---
%\usepackage{epstopdf}							% Konvertiert .eps zu .pdf on-the-fly (graphicx notwendig!)
%\usepackage{pdfpages}							% PDF Seiten einbinden


% --- Quellenangabe
\usepackage[autostyle]{csquotes}				% Anführungszeichen für deutsche Sprache und Quellenverzeichnis
\usepackage[%
	backend			= biber,					% bibtex oder biber [default]
	style			= numeric,    				% Literaturverzeichnisstil
	citestyle		= authortitle-icomp,		% Zitierstil
	sorting			= nty,						% Sortierung (Name, Titel, Jahr)
	natbib			= true,						% Kompatibilität mit Natbib-Bibliothek
	isbn			= false,					%
	url				= true,						%
	doi				= false,					%
	eprint			= false,					%
	block			= space,            		% kleiner horizontaler Platz zwischen den Feldern
	backrefstyle	= three+,       			% fasst Seiten zusammen, z.B. S. 2f, 6ff, 7-10
	date			= short,           			% Datumsformat
	bibwarn			= true,						%
	texencoding		= auto, 					% auto-detect the input encoding
	bibencoding		= auto 						% (auto (equal to tex), <encoding>)
]{biblatex}
%\usepackage[
%	square,
%	numbers
%]{natbib}										% Zitierstile (NUR für bibtex!)


% --- Zusatzpakete ---
\usepackage{blindtext}							% Ermöglichst Benutzung von Blindtext
\usepackage{tikz}
\usepackage{algorithm}							% Ermöglicht Erstellung von Algorithmen
\usepackage{algpseudocode}

% --- Verweise ---
\usepackage[hidelinks]{hyperref}				% Ermöglicht Verlinkungen [Immer als letztes!]